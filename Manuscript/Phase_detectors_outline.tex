\documentclass{article}
\title{Phase paper outline}
\author{}
\date{}
\begin{document}
\maketitle

\section*{Destination journal}
We would like to submit to an engineering journal. The reason is that we are not developing a \textit{seminal} work, just applying existing methods to multiple new applications. The journal will probably be a non-glamourous IEEE, a general physiology journal, or physiological methods journal. 

\subsection*{Potential physiology journals}
The paper should have this structure:\cite{muller_estimation_2003}
\begin{itemize}
\item Acta physiologica
\item Physiological Reports
\item AJP Heart/regulatory - Innovative methodology
\item The Journal of Physiology - techniques for physiology \textit{I think this kind of thing might be best}
\item The Journal of General Physiology - methods and approaches
\item Physiological measurement
\item Journal of experimental physiology
\item Journal of Applied Phyiology
\item E-Life
\end{itemize}

\subsection*{Potential engineering journals}
\begin{itemize}
\item IEEE transactions on biomedical engineering
\item IEEE transactions on signal processing
\item IEEE signal processing letters
\item Signal processing and control
\item Signal processing
\item Annals of biomedical engineering
\end{itemize}


\section*{Purpose of extracting phase information}

\subsection*{Instantaneous phase}
We are looking for the interaction between two oscillating physiological systems. Either we are looking for the relationship between two mutually synchronized oscillators, \textit{e.g.} renal autoregulation synchronization, or forced unidirectional synchronization \textit{e.g.} postural control indices and blood pressure. 

The way these interactions are detected is by phase difference, or by phase relationships over time. 

In order to get phase relationships, we need to extract instantaneous phase. There are multiple methods for determining instantaneous phase, but it is not clear which one is best, or even if there is a general "best" method.
\begin{enumerate}
\item Hilbert transform
\item Wavelet transform
\item Short time Fourier transform
\item Peak detection
\end{enumerate}

We will use these methods to determine when to use each particular method, and whether or not each should be used at all.

\subsection*{Frequency domain phase}

We should investigate the purposes of average phase. This is most likely to find signatures of regulation between one variable and another, or to find time delays. Regulation of one variable by another can result in a positive phase while time delays can cause negative phase.

\section*{When each method can fail}
\begin{itemize}
\item Frequency domain phase difference can fail when one oscillation leads another by more than half of a cycle.
\item Time-domain phase is only accessible at one frequency at a time if the Hilbert transform is used
\item Time-domain phase must have \textit{some} problems if wavelet. One big one is noise, and Hilbert is more resilient in the presence of noise.
\end{itemize}

\section*{Introduction}
\begin{itemize}
\item Physiological regulatory systems are reliably non-linear in their responses
\item Signal transduction, be it physical signals like pressure or cellular signals like phosphorylation, introduces a delay between input and output. 
\item Non-linearities and delays \cite{ottesen_modelling_1997} conspire to generate oscillations in physiological control systems, for example oscillations in renal blood flow, or oscillations in cerebral blood flow.
\item Oscillating physiological systems can become synchronized if they are somehow physically coupled. For example, sinus arrhythmia caused by respiration. 
\item Phase information extracted from each oscillating system can provide information about the relationship between the two systems.
\item Phase is essentially how far along a cycle a particular oscillation is. 
\item The problem is that phase has been extracted by multiple different methods, and it is unclear whether there is a best option, or if specific methods are ideal for specific purposes.
\item Noise and time-variance present in physiological data makes this process more difficult.

\item In addition to finding the individual phase of an oscillating signal, determination of phase shift, or lag-time, between two related signals is an important physiological parameter that may offer insight into pathophysiological conditions.  
\item The goal of this paper is to identify the most accurate methods of extracting phase information and determine how to deal with common hurdles present in physiological data.
\end{itemize}

Homeostasis is maintained by numerous interacting physiological control systems. Physiological systems are responsible for controlling a given output variable in response to one or more input variables. For example Heart rate is controlled by the autonomic nervous system to maintain blood pressure. Nonlinear responses between input and output in physiological systems are ubiquitous. Furthermore there are inherent delays in the response of each system because of inter and intra cellular signalling events. These nonlinearities

\section*{Methods}

\subsection*{Phase detectors}
\begin{itemize}
\item Peak detection
\item Autoregressive modelling
\item Short time Fourier transform
\item Hilbert transform \cite{muller_estimation_2003}
\item Wavelet transform \cite{farge_wavelet_1992,torrence_practical_1998}
\end{itemize}


\subsection*{Simulation}
\begin{itemize}
\item Generation of data
	\begin{itemize}
	\item A sinusoidal signal containing (number?) frequencies were generated to resemble a typical arterial pressure recording. 
	\item A second sinusoidal signal was generated so that it was the same as the original signal except each frequency was time-varying within a narrow range.
	\item Noise was added to the signal at increasing levels so SNR ranged from $1-100 dB$
	\item Each frequency was isolated with a bandpass filter (elliptic?)
	\end{itemize}
\item Testing the phase detectors
	\begin{itemize}
	\item Each phase detector was used to determine the instantaneous phase of each frequency band isolated from the simulated data
	\item The accuracy of the estimated instantaneous phase will be calculated using Bland-Altman statistics. The closet mean to zero wins. Techniques are deemed as broken when zero is not within $\overline{u}\pm1.96\sigma$
	\end{itemize}
	
\subsection*{Accuracy of phase estimates}
Accuracy $A$ is the mean difference between the estimated phase of a signal and its true phase. 
$$ A = \frac{1}{N} \sum_{m=1}^{N} \frac{\left| \phi _{true} (t) - \phi _{est} (t) \right|}{\phi _{true} (t)} $$

\subsection*{Physiological data}
\begin{itemize}
\item Renal cortical perfusion
	\begin{itemize}
	\item phase synchronization among regions of the kidney
	\end{itemize}
\item Blood pressure and heart rate
	\begin{itemize}
	\item Phase lag between changes in blood pressure and changes in heart rate
	
<<<<<<< HEAD
The oscillations of blood pressure and heart rate are often compared using cross-spectral analyses to quantify baroreflex function.  With this method, the coherence, transfer function gain, and phase relationship between the two signals are quantified.  The phase shift between blood pressure and heart rate is lengthened in some pathophysiological conditions (Javorka et al, 2011) and more variable/unstable in other conditions (Halamek et al, 2003 Circulation; Stewart et al., 2000).  Therefore, the phase relationship between these two signals is an important outcome measure for cardiovascular testing.  

=======
	\item The oscillations of blood pressure and heart rate are often compared using cross-spectral analyses to quantify baroreflex function.  With this method, the coherence, transfer function gain, and phase relationship between the two signals are quantified.  The phase shift between blood pressure and heart rate is lengthened in some pathological conditions \cite{tonhajzerova_cardiac_2011} and more variable/unstable in other conditions (\cite{halamek_variability_2003}; Stewart et al., 2000).  Therefore, the phase relationship between these two signals is an important outcome measure for cardiovascular testing.  
>>>>>>> 827626432738cf61b46be3d1187b6b7160e14265
	\end{itemize}
\item Something about SCG or ECG ?
\item Cerebral blood flow ?
\end{itemize}

\end{itemize}

\section*{Results}

	\subsection*{Figures}
	\begin{enumerate}
	\item Simulation
		\begin{enumerate}
		\item Time series of raw simulated data
		\item Time-domain phase 
		\item Accuracy of each detector at a range of signal to noise ratios
		\end{enumerate}
	\item Rat data
		\begin{enumerate}
		\item BP and cortical perfusion
		\item Phase interactions between two or three regions on the renal surface
		\end{enumerate}
	\item{Human data}
		\begin{enumerate}
		\item BP and heart rate
		\item Use phase to find the lag between a change in bp and a change in heart rate
		\end{enumerate}
	\end{enumerate}
	
	\subsection*{Tables}
		\begin{enumerate}
			\item Simulation
			\begin{enumerate}
			\item Noise conversation
				\begin{itemize}
					\item Add gaussian noise from snr=$\infty$ up to when it breaks down.
					\item It breaks down when 0 is not in the 95\% confidence interval of Bland-Altman-style statistics.		
				\end{itemize}
			\item Chirp (Frequency from point A to point B)
				\begin{itemize}
					\item Specifically to talk about limits to the bandpass filtering needed for the Hilbert method.
				\end{itemize}
			\item Length of signal
				\begin{itemize}
					\item Shorten the signal until techniques breakdown.
					\item Relative to frequency.
				\end{itemize}		
			\end{enumerate}
			\item Physiological Data
				
		\end{enumerate}
		
\bibliographystyle{plain}
\bibliography{OutlineBib}













\end{document}
